\chapter{Motivation and Contracts in Other Languages}

When thinking of developing new software, people think of using those software development tools and methods that will result in an increased overall productivity for the greater benefit. In the object-oriented world, productivity benefits are not just the result of the correct approach but also depend on how much emphasis is given to quality \cite{meyer1998building}.
Quality of a software system depends primarily on how reliable the software system is \cite{meyer1998building}. 
In the object-oriented world, a reliable piece of code is given an extra importance because of its reusable nature. Reusability is an important property of object-oriented programs, which will lose its relevance if the piece of code to be reused is not reliable and correct. 
Java is one of the most widely used object-oriented programming language that is used for commercial software development \cite{WhyisJav37:online}. There are different ways such as static typing and automatic garbage collection in Java that help ensure reliability. But this is not enough and we still need a better approach towards developing reliable software systems using Java. This forms the motivation behind the need of a contract system in Java.
This chapter focuses on different existing implementations of contract programming in Java and other languages along with their examples.

\section{Existing Implementations}

This section of the chapter lists a few existing implementations of contract programming available in different programming languages. 


\subsection{Contracts in Racket}

Contracts in the Racket programming language are mainly applied at module boundaries \cite{RacketContracts}. Thus, contract constraints and promises are imposed on the values that are exported from the module. Contracts in Racket can be attached to a definition or a function using the \tt{provide} keyword \cite{RacketContracts}.

\begin{minipage}{\linewidth}
Listing ~\ref{lst:racketcontractexample} shows an example that illustrates a basic contract in Racket.       
\begin{lstlisting}[frame=single, caption={Basic contract example in Racket},label = {lst:racketcontractexample}, captionpos=b, breaklines=true]
#lang racket
(provide (contract-out [amount positive?])) 
(define amount ...)
\end{lstlisting}
\end{minipage}

This specification in the Listing \ref{lst:racketcontractexample} states that the value of the \tt{amount} variable should always be positive. Every time the client refers to the amount, Racket's contract system keeps a check on the validation of the specified contract. If at some point the amount is bound to a non-positive number or to some value which is not a number, the contract system will signal a contract violation and blame the module breaking the promise \cite{RacketContracts}. Listing ~\ref{lst:RacketContractViolation} shows an example where the contract in Listing \ref{lst:racketcontractexample} fails and contract violation error will be thrown. \linebreak

\begin{minipage}{\linewidth}      
\begin{lstlisting}[frame=single, caption={Contract violation example in Racket}, label={lst:RacketContractViolation}, captionpos=b, breaklines=true]
#lang racket
(provide (contract-out [amount positive?])) 
(define amount 0)
\end{lstlisting}
\end{minipage}

Racket also allows contracts to be attached to functions \cite{RacketContractsFunc}. Contracts for functions in Racket are specified using the -> notation. 
\begin{minipage}{\linewidth}
Listing \ref{lst:FunctionContractRacket} gives an example of a bank module in which contracts are applied over functions of the module using the -> notation.       
\begin{lstlisting}[frame=single, caption={Contract over functions in Racket}, label={lst:FunctionContractRacket}, captionpos=b, breaklines=true]
#lang racket
(provide (contract-out
          [deposit (-> natural-number/c any)]
          [balance (-> natural-number/c)]))

(define amount 0)
(define (deposit a) (set! amount (+ amount a)))
(define (balance) amount)
\end{lstlisting}
\end{minipage}

In this example, contracts are applied over two functions : \tt{deposit} and \tt{balance} using the -> notation. The contract specified here states that \tt{deposit} is a function which accepts a non-negative integer and returns some value that is not specified in the contract. And, \tt{balance} is a function that takes in no argument and returns a non-negative integer.

\subsection{The Java Modeling Language}
The Java Modeling Language (JML) is a behavioral interface specification language for Java modules \cite{leavens2006design}. It provides set of some basic constructs that can be used to write contracts, in precondition and postcondition style for Java programs. Contracts in JML are provided using special annotation comments and are part of the Java code \cite{leavens2006design}. This contract definition, written in the form of comments  are converted into an executable code by the compiler. Thus, if any violation is detected while the code is executing, it can be immediately detected. Simple example of how contracts are written in Java using JML is given in Listing \ref{lst:JMLContract}  \cite{JavaMode99:online}. \linebreak

\begin{minipage}{\linewidth}      
\begin{lstlisting}[frame=single, language=Java, caption={JML Contract Example\cite{JavaMode99:online} }, label={lst:JMLContract}, captionpos=b, breaklines=true]
//@ requires 0 < amount && amount + balance < MAX_BALANCE;
//@ assignable balance;
//@ ensures balance == \old(balance) + amount;
public void credit(final int amount)
{
  this.balance += amount;
}
\end{lstlisting}
\end{minipage}

In this example, lines starting with the characters \tt{//@} denote a JML notation, which will be picked up by compiler and converted into executable code for assertions.
JML uses a \tt{requires} clause to implement a precondition and an \tt{ensures} clause to implement a postcondition.
The precondition in the ~\ref{lst:JMLContract} states that the amount should be greater than zero and the sum of the amount and the balance should remain less than the MAX\_BALANCE allowed.
The postcondition in the example given above states that the balance result should be equal to the sum of the old balance and the amount.
In JML, contracts can also be specified in the form of functions. Example given in \ref{lst:JMLFunctionContract} illustrates this type of contract \cite{leavens2006design}. \linebreak
 
\begin{minipage}{\linewidth}     
\begin{lstlisting}[frame=single, language=Java, caption={JML Contract Example with JML function \cite{leavens2006design} }, label={lst:JMLFunctionContract}, captionpos=b, breaklines=true]
package org.jmlspecs.samples.jmltutorial;
import org.jmlspecs.models.JMLDouble;
public class SqrtExample {
public final static double eps = 0.0001;
//@ requires x >= 0.0;
//@ ensures JMLDouble.approximatelyEqualTo(x, \result * \result, eps);
public static double sqrt(double x) {
return Math.sqrt(x);
}
}
\end{lstlisting}
\end{minipage}


In the listing \ref{lst:JMLFunctionContract}, the postcondition is specified with the help of a function \tt{approximatelyEqualsTo}. \tt{approximatelyEqualsTo} function checks if the produced result multiplied by itself is approximately equal to the input parameter x. If not, the code will fail the postcondition, causing an exception in the code execution. \linebreak
Thus, JML provides set of constructs and tools that allow a Java programmer to specify Eiffel-like contracts for their Java code. On similar lines, my project provides systematic approach of writing contracts in Java using custom annotations. However, there are no extra compilation steps like in the case of JML, that one needs to use. Contracts are execution ready as they are defined using Java annotations unlike JML where they are specified using Java comments.

\subsection{Contracts.js for JavaScript}

Contract.js is a library for JavaScript that provides a way to implement a higher-order behavioral contract system. It uses Sweet.js \cite{Contract11:online} and lets JavaScript programmers write contracts that dictate how exactly the program should behave. This library implements a runtime check for validity of contracts; if fault occurs, it pinpoints the exact section of the code that caused the failure with a descriptive message \cite{Contract11:online}.
Syntax for writing contracts using Contract.js. is shown in Listing \ref{lst:contractjssyntax}  \linebreak

\begin{minipage}{\linewidth}      
\begin{lstlisting}[frame=single, language=Java, caption={Contract.js syntax}, label={lst:contractjssyntax}, captionpos=b, breaklines=true]
@ (...) -> ...
function name(...) {
    ...
}
\end{lstlisting}
\end{minipage}


\begin{minipage}{\linewidth}       
\begin{lstlisting}[frame=single, language=Java, caption={Contract using Contract.js},label={lst:contractjscontract}, captionpos=b, breaklines=true]
@ ({age: Num}) -> Bool
function isAdult(o) {
    return o.age > 18;
}    
\end{lstlisting}
\end{minipage}

Listing \ref{lst:contractjscontract} shows an example of a contract written for a JavaScript function using Contract.js. In this example, there are two parts to the contract specified for the function \tt{isAdult}. The first part states that any object o passed as an input parameter to the \tt{isAdult} function should have a property named \tt{age} with a value of type \tt{Num}. In the second part of the contract, a valid input function is obligated to return a \tt{Bool} value at the end of the execution. Here we can relate the first part of the contract as a precondition and the second part of the contract as a postcondition. If the program fails to satisfy any of these two, the program will exit with a well defined error message.\linebreak

\begin{minipage}{\linewidth}
\begin{lstlisting}[frame=single, language=Java, caption={Example of a failed contract}, label={lst:FailedJSContract}, captionpos=b]
isAdult({
    name: "John",
});    
\end{lstlisting}
\end{minipage}
Listing \ref{lst:FailedJSContract} shows an example of an error message that is generated when a contract fails to hold true.
Here, the \tt{age} property is missing from the object that is passed in as an input parameter to the \tt{isAdult} function, which results in a failed contract. Listing \ref{lst:failedcontracterror} shows an error that code in Listing \ref{lst:FailedJSContract} generates.
\linebreak

\begin{minipage}{\linewidth}
\begin{lstlisting}[frame=single, language=Java, caption={Error message on failed contract}, label={lst:failedcontracterror}, captionpos=b, breaklines=true]
Error: isAdult: contract violation
expected: Num
given: undefined
in: the age property of
    the 1st argument of
    ({age: Num}) -> Bool
function isAdult guarded at line: 2
blaming: (calling context for isAdult)

\end{lstlisting}
\end{minipage}

Another example of a failed contract with respect to the contract in Listing \ref{lst:FailedJSContract}, where \tt{name} property exists but with an invalid type, is given in Listing \ref{lst:failedcontract2}. \linebreak

\begin{minipage}{\linewidth}
\begin{lstlisting}[frame=single, language=Java, caption={Example of a failed contract}, label={lst:failedcontract2}, captionpos=b, breaklines=true]
isAdult({
    name: "John",
    age: "Five"
});

Error: isAdult: contract violation
expected: Num
given: Str
in: the 1st field of
    the age property of
    the 1st argument of
    ({age: Num}) -> Bool
function isAdult guarded at line: 2
blaming: (calling context for isAdult)
   
\end{lstlisting}
\end{minipage}

From these examples, it is clear that the contract system enforced by Contract.js is for checking type related errors that may occur in the code. It also provides number basic contracts that check for first order properties \cite{Contract11:online}.    


\section{Contract Library for Java by Neha Rajkumar}
This Java library \cite{rajkumar2015designing} developed by Neha at SJSU under the guidance of Dr. Thomas Austin uses custom annotations and AspectJ to implement contract system in Java. This library provides a custom Java annotation "@contract" using which a developer can provide preconditions and postconditions over a Java method. These pre and postconditions constraints are then checked at runtime for their validity using AspectJ and reflection. I will be extending this approach towards building my library.   	     