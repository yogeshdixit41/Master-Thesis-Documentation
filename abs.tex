Today many large and complex software systems are being developed in Java. Although, software
always had bugs despite these bugs it is very important that these developed 
systems are more reliable. 
\paragraph{}
One way that we can help achieve this is the Design by Contract
(DbC) paradigm, which was first introduced by Bertrand Meyer, the
creator of Eiffel.
The concept of DbC was introduced for software developers so that they
can produce more reliable software systems with a little extra cost.
Using programming contracts allows developer to specify details
such as input conditions and expected output conditions. Doing this
makes it easy for the system to assign blame whenever software runs
into some erroneous state. Once the blame is assigned it is easier for
the developer to detect the cause, so that the appropriate actions can
be taken to resolve the issue. 
\paragraph{} 
My project develops a library in Java that allows 
developers to write contracts for their Java programs. These contracts should be written using a custom annotation "@contract", which will have pre-conditions and post-conditions as it's attributes. These contracts are then evaluated
by the library with the help of Prolog dictionary which acts as the database. 
\paragraph{}
Prolog is widely known for its implementations in the area of Artificial Intelligence and Natural Language Processing. In my implementation of contract programming, a Prolog file/s that contain set of facts and rules will form the basis of contract evaluation. To use Prolog for this purpose was first suggested to me by my mentor Dr. Thomas Austin. After analyzing and understanding its basic constructs, I found that it is very easy and efficient to write the set of rules using Prolog that you as a developer want your contracts confine to. Once you have these rules in place, my library will query the Prolog file to evaluate the validity of the contract conditions.
\paragraph{}
In this paper I will provide an example of how my library works with a sample application of Quicksort program. I will also talk about different performance tests and its results carried out on this sample application.       