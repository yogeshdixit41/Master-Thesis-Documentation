Nowadays, there are large and complex software systems that already
exist or are being developed in almost all the different industries and
sectors. A programming language used for developing a large share
of these software systems is Java. Although it is true that software
always had bugs and always will, it is very important that these developed 
systems are more reliable. There are many different ways in
which we can achieve this, one of which is the Design by Contract
(DbC) paradigm, which was first introduced by Bertrand Meyer, the
creator of Eiffel (an object-oriented programming language).
Concept of DbC was introduced for software developers so that they
can produce more reliable software systems with a little extra cost.
Using programming contracts allows developer to specify the details
such as input conditions and expected output conditions. Doing this
makes it easy for the system to assign blame whenever software runs
into some erroneous state. Once the blame is assigned it is easy for
the developer to detect the cause, so that the appropriate actions can
be taken to resolve the issue. My project aims at developing a library (written in Java), using which 
developers can write contracts for their Java programs. These contracts will be evaluated
by my library with the help of Prolog dictionary which acts as the database of facts. Using this contract system, developers would be able to increase the robustness of their software written in Java.
