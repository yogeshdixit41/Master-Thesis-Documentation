Today many large and complex software systems are being developed in Java. Although, software
always has bugs, it is very important that these developed 
systems are more reliable despite these bugs. 
\paragraph{}
One way that we can help achieve this is the Design by Contract
(DbC) paradigm, which was first introduced by Bertrand Meyer, the
creator of Eiffel.
The concept of DbC was introduced for software developers so that they
can produce more reliable software systems with a little extra cost.
Using programming contracts allows developer to specify details
such as input conditions and expected output conditions. Doing this
makes it easy for the system to assign blame whenever software runs
into some erroneous state. Once the blame is assigned it is easier for
the developer to detect the cause, so that the appropriate actions can
be taken to resolve the issue. 
\paragraph{} 
My project develops a library in Java that allows 
developers to write contracts for their Java programs in Prolog. These contracts should be written using a custom annotation "@contract", which will have pre-conditions and post-conditions as its attributes. These contracts are then evaluated
by the library with the help of Prolog dictionary which acts as the database. 
\paragraph{}
Prolog's declarative style is a natural fit for writing contracts. With this project, I hope to simplify writing contracts for Java developers. In this paper, I review my implementation. I further discuss some performance tests to show the added overhead.       