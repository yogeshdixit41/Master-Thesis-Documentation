\chapter{Implementation}

This chapter focuses on the implementation details of my project and how to use the library for implementing contracts in Java.

\section{Java Custom Annotations}

Annotations in Java is an element which provides information or data about the data. It can be termed as a form of metadata which provides more information at run-time or compile-time about the part of the code that is being annotated. Java annotations always start with the \@ symbol and can be of different forms. Some annotations like \@override annotation don't have any elements whereas some annotations like \@SuppressWarnings("unchecked") come with a element defined inside the parentheses. 
Java also provides a way to define custom annotations type using \@interface. In my library I have used this method to create a custom annotation type "\@contract", using which developers will specify the contracts for their Java methods.
Below given is the code block for creating the custom annotation type \@contract.

\begin{minipage}{\linewidth}
\lstset{language=Java, caption=CustomAnnotationtypeJava, captionpos=b, breaklines=true, showstringspaces=false}       
\begin{lstlisting}[frame=single]

package annotations;

import java.lang.annotation.ElementType;
import java.lang.annotation.Retention;
import java.lang.annotation.RetentionPolicy;
import java.lang.annotation.Target;

@Target(ElementType.METHOD )
@Retention(RetentionPolicy.RUNTIME)

public @interface Contract {
		
		String [] pre_cond () default "";
		String [] post_cond() default "";
		String [] source_files() default "no file to load";
 }

\end{lstlisting}
\end{minipage}

Target element specifies where exactly the annotation type can be used in the code \cite{JavaAnnotationsTutorial}. In the above implementation it specifies that \@contract annotation type can be used only with methods. 
\@Retention element specifies till what point in the execution cycle of the code should the annotation of this type be available \cite{JavaAnnotationsTutorial}. In case of \@contract, @Retention specifies that it will be made available till Runtime.   
\@Contract custom annotation type has 3 elements in it: pre\_cond, post\_cond and source\_files. All these three elements are of type arrays of Strings, that is each element can have multiple string values assigned when writing a contract. All these tree elements come with default values associated to it, which makes them non-compulsory elements of the contract type. Use of each of these elements is as follows:
\begin{itemize}
\item pre\_cond: This element is used to specify the preconditions of the contract.
\item post\_cond: This element is used to specify the postconditions of the contract.
\item source\_files: This element is used to specify the prolog files which should be referred to validate the preconditions and postconditions.
\end{itemize} 

Below given is an example which illustrates how a contract can be written using above defined \@contract custom annotation type.
\linebreak


\begin{minipage}{\linewidth}
\lstset{language=Java, caption=CustomAnnotationtypeJava, captionpos=b, breaklines=true}       
\begin{lstlisting}[frame=single]

@Contract(
pre_cond = { "isPositive(amount)", "lessThan(amount, @balance)" }, 
post_cond = { "checkbalance(ans)" }, source_files = { "bankprolog.pl" })

\end{lstlisting}
\end{minipage}

In the above example of a contract, isPositive(amount) and lessThan(amount, \@balance) are declared as preconditions, checkbalance(ans) is the postcondition. "bankprolog.pl" is a Prolog file that is specified as a source for validating the contract conditions.

\section{AspectJ and Reflection}

\section{Prolog for Contract Validation}

\section{} 
  