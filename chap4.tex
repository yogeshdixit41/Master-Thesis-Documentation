\chapter{Implementation}

This chapter focuses on the implementation details of my project and how to use the library for implementing contracts in Java.

\section{Java Custom Annotations}

Annotations in Java can be used to retrieve information or data about the data. It can be termed as a form of metadata which provides more information at run-time or compile-time about the part of the code that is being annotated. Java annotations always start with the \tt{@} symbol and can be of different forms. Some annotations like the \tt{@override} annotation do not have any elements whereas some annotations like \tt{@SuppressWarnings("unchecked")} come with a element defined inside the parentheses. 
Java also provides a way to define custom annotations using \tt{@interface}. In my library, I have used this method to create a custom \tt{@contract} annotation, which developers can use to specify the contracts for their Java methods.
Listing \ref{JavaCustomAnnotation} shows code block for creating the custom annotation type \tt{@contract}.

\begin{minipage}{\linewidth}      
\begin{lstlisting}[frame=single, language=Java, caption={Custom Annotation Type}, label={JavaCustomAnnotation}, captionpos=b, breaklines=true, showstringspaces=false]

package annotations;
import java.lang.annotation.ElementType;
import java.lang.annotation.Retention;
import java.lang.annotation.RetentionPolicy;
import java.lang.annotation.Target;

@Target(ElementType.METHOD )
@Retention(RetentionPolicy.RUNTIME)

public @interface Contract {
	String [] pre_cond () default "";
	String [] post_cond() default "";
	String [] source_files() default "no file to load";
 }

\end{lstlisting}
\end{minipage}


The \tt{Target} element specifies where the annotation type can be used in the code \cite{JavaAnnotationsTutorial}. In the implementation given in Listing \ref{JavaCustomAnnotation}, it specifies that the \tt{@contract} annotation type can be used only with methods. 
The \tt{@Retention} element specifies until what point in the execution cycle of the code should the annotation of this type be available \cite{JavaAnnotationsTutorial}. In the case of \tt{@contract}, \tt{@Retention} specifies that it will be made available until runtime.   
The \tt{@Contract} custom annotation has 3 elements: pre\_cond, post\_cond, and source\_files. All these three elements are string arrays; that is, each element can have multiple string values assigned when writing a contract. All these tree elements come with default values associated to them, which makes them non-compulsory elements of the contract type. Use of each of these elements is as follows:
\begin{itemize}
\item pre\_cond: This element is used to specify the preconditions of the contract.
\item post\_cond: This element is used to specify the postconditions of the contract.
\item source\_files: This element is used to specify the Prolog files which should be referred to validate the preconditions and postconditions.
\end{itemize} 

Listing \ref{ContractExample} gives an example that illustrates how a contract can be written using the previously defined  custom annotation type.
\linebreak


\begin{minipage}{\linewidth}     
\begin{lstlisting}[frame=single, language=Java, caption={using custom annotation to write contract}, label={ContractExample}, captionpos=b, breaklines=true, showstringspaces=false]

@Contract(
pre_cond = { "isPositive(amount)", "lessThan(amount, @balance)" }, 
post_cond = { "checkbalance(ans)" }, source_files = { "bankprolog.pl" })

\end{lstlisting}
\end{minipage}

In the Listing \ref{ContractExample}, \tt{isPositive(amount)} and \tt{lessThan(amount, \@balance)} are declared as preconditions, \tt{checkbalance(ans)} is the postcondition. "bankprolog.pl" is a Prolog file that is specified as a source for validating the contract conditions.

\section{AspectJ and Reflection}
Once the contract is specified over a method using the \tt{@contract} annotation, its conditions are validated at runtime. Specifically, preconditions are evaluated just before execution enters the method routine and postconditions should be evaluated immediately after the method has executed. AspectJ, an extension of Java, provides this exact granularity and control over the Java program.
AspectJ is an aspect-oriented programming extension created for the Java programming language \cite{AspectJWiki:online}. It provides \tt{pointcuts}, which specify well defined moments in the execution of a program, such as a method call \cite{AspectJWiki:online}. It also provides \tt{before()} and \tt{after()} routines which can be used for implementing precondition and postcondition contracts.
Listing \ref{ApsectJPointcut} pointcuts that I am using in this library. \linebreak

\begin{minipage}{\linewidth}       
\begin{lstlisting}[frame=single, language=Java, caption={Pointcut in AspectJ},label={ApsectJPointcut}, captionpos=b, breaklines=true, showstringspaces=false]
//pointcut to catch execution context of any method
pointcut f () : execution (* *(..) ) ;	

//pointcut to catch annotations from the code
pointcut g () : @annotation(Contract);
\end{lstlisting}
\end{minipage} 

\begin{itemize}
\item f : This pointcut is used to catch the execution context of the running Java program. It uses the wild card syntax "(* * (..))" which specifies that, this pointcut will pick the execution moment of any method in the executing Java program, irrespective of its signature. 

\item g : This pointcut specifically checks for the elements in the executing Java program that are annotated by the Contract annotation type.
\end{itemize}

Composing these two pointcuts using the \&\& operation makes it possible to achieve a pointcut that will be picked up only when a method annotated with the Contract annotation is called for execution. \linebreak

\begin{minipage}{\linewidth}       
\begin{lstlisting}[frame=single, language=Java, caption={Pointcut Composition and before-after routines}, label={beforeafteroutines}, captionpos=b, breaklines=true, showstringspaces=false]
// Joining both the pointcuts it will catch 
//annotations in the execution context
before () : f() && g() 
{
	...
}
	
after () returning ( Object objret ): f () && g()
{
	...
}
\end{lstlisting}
\end{minipage} 


\section{Prolog for Contract Validation}

Prolog is widely known for its implementations in the area of Artificial Intelligence and Natural Language Processing. In my implementation of contract programming, Prolog files that contain set of facts and rules will form the basis of contract evaluation. After analyzing and understanding its basic constructs, I found that Prolog's declarative style can be efficiently used to write the set of rules and facts to specify contracts. Once you have these rules in place, the library will query the Prolog file to evaluate the validity of the contract conditions.

\subsection{Basic Prolog Constructs and Syntax}

Prolog has three basic constructs that I will be focusing in this part of the chapter \cite{LearnProlog:online}.

\begin{itemize}
\item Facts
\item Rules
\item Queries
\end{itemize}

\subsubsection{Facts}
A fact is a simple statement of the form " chinese (chow\_mein). " which results in true or false value. Given this fact, we can now ask \tt{ is chow\_mein a chinese dish ?} , which will return true. This can be done using Prolog Queries.

\subsubsection{Rules}
A rule is collection of one or more facts. Multiple facts in conjunction or dis-junction form the result of a rule. Rules are of the form \tt{nonnegative(Var):- Var >= 0. }. Using this rule, one can query and check if a number is positive or not.

\subsubsection{Queries}
Queries are an important construct of Prolog which allows us to ask questions to the Prolog engine and get answers from it. The Prolog  file contains one or more facts and rules based on which our queries will be answered.
Listing \ref{PrologQueryExample} gives an example of a simple Prolog code and some associated queries. \linebreak

\begin{minipage}{\linewidth}    
\begin{lstlisting}[frame=single, caption={Prolog Queries}, label={PrologQueryExample}, captionpos=b, breaklines=true, showstringspaces=false]
likes(sam,Food) :-
        indian(Food),
        mild(Food).
likes(sam,Food) :-
        chinese(Food).
likes(sam,Food) :-
        italian(Food).
likes(sam,Food) :-
	spanish(Food).
likes(sam,chips).

spanish(chicken_chillie).
indian(chicken_curry).
mild(chicken_curry).
chinese(chow_mein).
italian(pizza).
italian(spaghetti).

\end{lstlisting}
\end{minipage} 

For the above Prolog program we can formulate different queries as shown in Listing \ref{PrologProgQueries},

\begin{minipage}{\linewidth}       
\begin{lstlisting}[frame=single, caption={Prolog example}, label={PrologProgQueries}, captionpos=b, breaklines=true]
?- likes(sam,What).
What = chicken_curry ;
What = chow_mein ;
What = pizza ;
What = spaghetti ;
What = chicken_chillie ;
What = chips.

?- likes(sam,pizza).
true
?- likes(sam,burger).
false.
?- italian(chicken_chillie).
false.
?- chinese(chow_mein).
true.

\end{lstlisting}
\end{minipage}

\subsection{Prolog for Contracts}

Using the facts and rules introduced in the section above we can specify contract rules effectively. Developers can write their own Prolog files to create custom contracts. For instance, a developer might write a postcondition contract for a quicksort partition function. This contract needs to validate that at each iteration a valid pivot is chosen. This contract can be written using custom annotation like this: \tt{@Contract( post\_cond = { "checkPivotValid(ans,@arr)." })}. 
Once this is done, the developer has to think on the logic that \tt{checkPivotValid} should follow. Here, the logic would check if the selected pivot element is greater than all the left elements and less than all the right elements at each iteration. Once this logic is decided, it can be easily converted into a Prolog rule. Listing \ref{PrologExample} shows the Prolog code for this contract.

\begin{minipage}{\linewidth}       
\begin{lstlisting}[frame=single, caption={Prolog example}, label={PrologExample} captionpos=b, breaklines=true]
sublist(S,M,N,[_A|B]):- 
	M > 0, 
	M < N, 
	sublist(S,M-1,N-1,B).	
sublist(S,M,N,[A|B]):- 
	0 is M, 
	M<N, 
	N2 is N-1, 
	S=[A|D], 
	sublist(D,0,N2,B).	
sublist([],0,0,_).

checkPivotValid(Pivotindex,List):-
	Pindex is Pivotindex,
	sublist(S,0,Pindex,List),
	nth0(Pivotindex,List,Pivotelement),
	check_prelist_util(S,Pivotelement),
	countElements(List,Count),
	sublist(N,Pindex+1,Count,List),
	check_postlist_util(N,Pivotelement).
\end{lstlisting}
\end{minipage}

\begin{minipage}{\linewidth}
\begin{lstlisting}[frame=single, caption={Prolog example}, captionpos=b, breaklines=true]
countElements([],0).	
countElements([_|Xs],Count):-
	countElements(Xs,Count1),
	Count is Count1+1.

check_prelist_util([H|T], N) :-
    H =< N,
    check_prelist_util(T, N).	
check_prelist_util([H|[]], N) :-
    H =< N.
	
check_postlist_util([H|T], N) :-
    H > N,
    check_postlist_util(T, N).	
check_postlist_util([H|[]], N) :-
    H > N.
\end{lstlisting}
\end{minipage}


checkPivotValid rule from the above Prolog code checks if the pivot selected is greater than all the elements to its left and less than all the elements to its right in the list.

\section{JIProlog}

JiProlog is a Prolog interpreter written in Java \cite{JIProlog}.As seen in above sections, rules for Java contracts are specified in a Prolog file. We need to evaluate these rules to check the validity of the contracts. For this, we need some way using which we can query the Prolog files from Java aspectj code.
JIProlog provides APIs using which we can establish this connectivity between a Java and Prolog code \cite{JIProlog}. Using these APIs we can submit a Prolog query from Java code and get the results. This solves the problem of evaluating contract rules from Java.

\begin{minipage}{\linewidth}
Listing \ref{JIPrologConnection} shows a code snippet that illustrates connection achieved between Java and Prolog code using JIProlog library APIs.
     
\begin{lstlisting}[frame=single, language=Java,caption={Creating JIPEngine instance}, label={JIPrologConnection}, captionpos=b, breaklines=true, showstringspaces=false]

public class JIPInitializer {
	public final JIPEngine jip = new JIPEngine(); 
	public JIPInitializer()
	{
		jip.setDebug(false);
		jip.setTrace(false);
		jip.setEnvVariable("debug", "off");
		try
		{
			jip.consultFile("default_prolog_library.pl");	
		}catch(JIPSyntaxErrorException ex)
		{
			System.out.println("Exception loading prolog file : " + ex.getMessage());
		    System.exit(0);
		}
	}
}
\end{lstlisting}
\end{minipage}

Above given code snippet illustrates how to create an instance of the JIPEngine class, which then will be used for loading a Prolog file and making API calls. JIPEngine is the main class of the JIProlog library \cite{JIProlog}. It supplies all the methods required for the Java-Prolog connection and making queries. Code snippet in given in Listing \ref{CodeSnippet} illustrates how to load a Prolog file using JIPEngine instance. \linebreak

\begin{minipage}{\linewidth}       
\begin{lstlisting}[frame=single, language=Java,caption={Loading a Prolog file using JIProlog API}, label={CodeSnippet}, captionpos=b, breaklines=true, showstringspaces=false]
public void loadFile(String fileName)
	{
		try
		{
			jip.consultFile(fileName);
		}
		catch(JIPSyntaxErrorException ex)
		{
			System.out.println("Exception loading prolog file : " + ex.getMessage());
			System.exit(0);
		}
	}
\end{lstlisting}
\end{minipage}

JIPEngine's consultFile method compiles and loads the Prolog file, whose name is passed to it as parameter. Once the file is loaded it is ready to be queried using query API's. There are two ways to submit Prolog queries using JIPEngine: Synchronously and asynchronously \cite{JIProlog}. For my library I have used synchronous API calls. 

\begin{minipage}{\linewidth}
Listing \ref{PrologQueryJIProlog} shows how we can submit a query using JIPEngine instance and read the solution of the submitted query.
      
\begin{lstlisting}[frame=single, language=Java, caption={Prolog Query Using JIProlog API}, label={PrologQueryJIProlog}, captionpos=b, breaklines=true, showstringspaces=false]
JIPQuery jipQuery = jip.openSynchronousQuery(queryString);
boolean queryResult = readSolution(jipQuery);
\end{lstlisting}
\end{minipage}

JIProlog makes it very easy, querying Prolog rules and facts within Java scope and forms the connecting piece of the contract library. Next chapter will focus on a contract example for a QuickSort program and its performance results.

 